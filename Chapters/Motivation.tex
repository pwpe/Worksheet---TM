
Approximately 1300 patients are diagnosed with bladder cancer in Denmark each year \citep{bladderregristry}. The Department of Urology at Aarhus University Hospital (AUH) in Aarhus, Denmark is highly specialized in bladder cancer and performed 123 cystectomies (surgical procedure for urinary bladder removal) in 2014. 90 \% of these surgeries were combined with a urostomy \citep{louise} - a  surgically made opening in the belly (stoma) to re-direct the urine away from the bladder, collected in a pouch worn by stoma patients. 

A stoma surgery has great impact on a patient's daily life, both physically and psychologically. Previous studies have identified that a patient's ability to self-manage their stoma is an essential factor in determining positive psychological adjustments to life after a stoma surgery \citep{piwonka}. Teaching patients self-management skills after a stoma surgery is a critical aspect of stoma care, but also in terms of improving patients' quality of life  \citep{piwonka, white1997}. 

Today, it is the hospital-based stoma nurse, who teaches the patients self-management skills after a surgery. However, time available in-hospital to teach patients stoma care after a cystectomy has significantly decreased in Denmark \citep{jensen_validation}, which has created an obvious demand for effective low-cost educational interventions that aims to improve quality of care in the short time available. One such intervention is the Urostomy Education Scale (UES) that has recently been invented with the purpose of reducing random clinical practice among nurses and improve quality of care, when teaching patients self-management skills. However, despite of the many benefits the UES has brought to current practice, it has also introduced some new challenges.

This project aims to give an insight into the challenges experienced in current practice due to the introduction of the UES and how these challenges could be solved by the design and implementation of a mobile application. It further emphasizes the importance of improving quality of care for stoma patients and future perspectives on, how mobile technology can be used to do that.  

%that makes it a time consuming process for nurses to use the intervention on patients. This is a problem, because the time available in-hospital to teach patients self-management of their stoma after a cystectomy has significantly decreased in Denmark \citep{jensen_validation}. 

%Many patients experience loss of physical and emotional control after a stoma surgery \citep{metcalf, bekkers}. 




%However, 	

%The Urostomy Education Scale (UES) is the first standardized educational intervention 

%Today, it is the hospital-based stoma nurse, who educates the patient's in 

%According to experts, a standardized validated and evidence-based educational intervention can improve quality of stoma care and reduce random clinical practice. The introduction of the Urostomy Education Scale (UES), the first standardized educational intervention that makes it possible to document a patient's stoma self-care skills, was found to save time and make it easier for nurses to manage the process of teaching a patient stoma self-care. However, the systems used in current practice do not facilitate nurses in teaching stoma self-care to patients. Challenges include missing overview of patient's progression, portability and availability.

%Previous studies found that 

%The hospital-based stoma care nurse spe- cialist is ideally placed to assist the patient through this period of adaptation to enable successful rehabilitation to take place.


%A urostomy is a surgically made opening in the belly to re-direct the urine away from the bladder. The opening on the belly, also called a \textit{stoma}, makes it possible to pass out urine to a pouch that stoma patients wear. 

%Health Information Technology 
%\begin{itemize}
%\item Lowering health care cost 
%\item Reducing medical errors 
%\end{itemize}
%Smartphones and HIT
%Market and trend analysis 
