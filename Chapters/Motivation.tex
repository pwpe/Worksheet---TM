In recent years, the proportion of chronically ill patients has increased in accordance with an expanding elderly population \citep{motWHO}. By 2030 the World Health Organization (WHO) has predicted that Chronic Obstructive Pulmonary Disease (COPD) will affect over 64 million people and will be the third leading cause of death worldwide \citep{motStat}. 

COPD is a progressive lung disease in which the airways are damaged, resulting in periods of exacerbations among those affected. An exacerbation can be defined as \textit{"a sustained worsening of the patient's condition, from the stable state and beyond normal day-to-day variations, that is acute in onset and necessitates a change in regular medication in a patient with underlying COPD"} \citep{motExace}, where the worsening of the patient's condition refers to a worsening in symptoms or lung function. Patients with COPD experience symptoms, such as shortness of breath (known as dyspnea), phlegm, cough, wheezing and chest tightness \citep{motSymp}. 
 
 Research shows that delayed treatment following the onset of an exacerbation results in increased use of healthcare services, readmission to the hospital and a decline in health-related quality of life \citep{motExace}. Early detection and treatment of exacerbations are therefore an important step towards preventing decline in patients' quality of life and healthcare utilisation.  



%[1] Global Health and Aging, World Health Organization & National Institute on Aging and National Institutes of Health, 2011.
%
%[2] World Health Organization. Department of Measurement and Health Information Systems of the Information, Evidence and Research Cluster. 2008. World Health Statistics 2008. World Health Organization.
%
%[3] Kessler, R., Partridge, M. R., Miravitlles, M., Cazzola, M., Vogelmeier, C., Leynaud, D., & Ostinelli, J. (2011). Symptom variability in patients with severe COPD: a pan-European cross-sectional study. European Respiratory Journal, 37(2), 264-272.
%
%[4] Rodriguez-Roisin, R. (2000). Toward a consensus definition for copd exacerbations*. CHEST Journal, 117(5_suppl_2), 398S-401S.
%
%[5] Wilkinson, T. M., Donaldson, G. C., Hurst, J. R., Seemungal, T. A., & Wedzicha, J. A. (2004). Early therapy improves outcomes of exacerbations of chronic obstructive pulmonary disease. American journal of respiratory and critical care medicine, 169(12), 1298-1303.
%
%[6] Field, M. J. (Ed.). (1996). Telemedicine: A guide to assessing telecommunications for health care. National Academies Press.
%
%[7] Pedone, C., & Lelli, D. (2015). Systematic review of telemonitoring in COPD: an update. Pneumologia i Alergologia Polska, 83(6), 476-484.
%
%[8] Pennebaker, J. W. (2000). Psychological factors influencing the reporting of physical symptoms. The science of self-report: Implications for research and practice, 299-315.
%
%[9] Ure, J., Pinnock, H., Hanley, J., Kidd, G., Smith, E. M., Tarling, A., ... & McKinstry, B. (2011). Piloting tele-monitoring in COPD: a mixed methods exploration of issues in design and implementation. Primary Care Respiratory Journal, 21(1), 57-64.
%
%[10] Bestall, J. C., Paul, E. A., Garrod, R., Garnham, R., Jones, P. W., & Wedzicha, J. A. (1999). Usefulness of the Medical Research Council (MRC) dyspnoea scale as a measure of disability in patients with chronic obstructive pulmonary disease. Thorax, 54(7), 581-586.
%
%[11] Kendrick, K. R., Baxi, S. C., & Smith, R. M. (2000). Usefulness of the modified 0-10 Borg scale in assessing the degree of dyspnea in patients with COPD and asthma. Journal of Emergency Nursing, 26(3), 216-222.
%
%[12] Gift, A. G. (1989). Validation of a vertical visual analogue scale as a measure of clinical dyspnea. Rehabilitation Nursing, 14(6), 323-325.
%
%[13] Corbishley, P., & Rodríguez-Villegas, E. (2008). Breathing detection: towards a miniaturized, wearable, battery-operated monitoring system. Biomedical Engineering, IEEE Transactions on, 55(1), 196-204.
%
%[14] Hung, P. D., Bonnet, S., Guillemaud, R., Castelli, E., & Yen, P. T. N. (2008, May). Estimation of respiratory waveform using an accelerometer. In Biomedical Imaging: From Nano to Macro, 2008. ISBI 2008. 5th IEEE International Symposium on (pp. 1493-1496). IEEE.
%
%[15] Kawamoto, K., Tanaka, T., & Kuriyama, H. (2014, September). Your activity tracker knows when you quit smoking. In Proceedings of the 2014 ACM International Symposium on Wearable Computers (pp. 107-110). ACM.
%
%[16] Schlecht, N. F., Schwartzman, K., & Bourbeau, J. (2005). Dyspnea as clinical indicator in patients with chronic obstructive pulmonary disease. Chronic respiratory disease, 2(4), 183-191.
%
%[17] Okubo, M., Imai, Y., Ishikawa, T., Hayasaka, T., Ueno, S., & Yamaguchi, T. (2009). Development of automatic respiration monitoring for home-care patients of respiratory diseases with therapeutic aids. In 4th European Conference of the International Federation for Medical and Biological Engineering (pp. 1117-1120). Springer Berlin Heidelberg.
%
%[18] Colantonio, S., Dellacà, R. L., Govoni, L., Martinelli, M., Salvetti, O., & Vitacca, M. (2012). A Decision Making Approach for the Remote, Personalized Evaluation of COPD Patients’ Health Status. Proc. 7th Int. W. BSI, 2-4.
%
%[19] Chan, M., Estève, D., Fourniols, J. Y., Escriba, C., & Campo, E. (2012). Smart wearable systems: Current status and future challenges. Artificial intelligence in medicine, 56(3), 137-156.
%
%[20] Davis, F. D. (1989). Perceived usefulness, perceived ease of use, and user acceptance of information technology. MIS quarterly, 319-340.