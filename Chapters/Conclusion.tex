A stoma nurse has an important role, in educating a patient to adjust to a new lifestyle after a stome surgery. It is the stoma nurse at the hospital, who teaches the patient stoma-management skills. Stoma self-care is acknowledged as a crucial factor in determining a patient's quality of life \citep{piwonka}. 

In this project we focused on designing and implementing a tool using mobile technology to facilitate the nurses in teaching patients stoma self-care, more specifically changing their stoma appliance. The application was based on the paper-version of the Urostomy Education Scale (UES) and had the primary objectives of scoring the patient according to the UES standards and giving the nurses an overview of the patient's current progression. Through an iterative prototyping process and testing, we found that the application had potential in terms of supporting nurses in teaching patients self-management. 

In this project, it was acknowledged that more field studies and testing would benefit the application. Design and prototyping is still on-going and a high-fidelity test is expected to take place in Spring 2016 to further improve the application. 



%User needs in context study would have been preferable 
%- Kortlægge arbejdsgangen og nuværende interaktion mellem patient og nurse 
%- Hvilke behov har nurse? Hvad bruger man tallene til i dag? 
%- Hvilke behov har patienten? 
%- Tilbage/fremgang vigtig eller forventet?


%
%Correct the design and make another iteration 
%- Prepare for pilot testing in Spring 2016


\section{Future Perspectives}
While the UES is an intervention that primarily focuses on the post-operative phase, it is important to emphasize that introducing the patient for education and counseling about e.g. stoma self-care already in the pre-operative phase is recommended. 

This project dealt with design and implementation of a tool to facilitate nurses in educating patients to change their stoma appliance. Stoma self-care also involves other topics, such as identifying warning signs and managing such signs. In the process of teaching patients stoma self-care, much of the responsibility in assessing the patient's individual needs and motivating the patient is still left to the nurse. 

Another perspective on this matter, could be empowering the patients (already in the pre-operative phase) to self-manage their stoma appliance and other stoma-related issues. This perspective could be considered to further improve the quality of life of stoma patients. An application for stoma patients should take into account that motivation is an important prerequisite for learning and the application should support patient's individual needs. Access to relevant information should be prioritized, but focus should also be on preparing the patient to change behavior and adjust his/her lifestyle. This perspective requires identifying users needs in the context through field studies, workshops, interviews and similar user-centered design methods. 

Further consideration on including the community nurses, physicians, patients and other important stakeholders (e.g. relatives) in the use of the current application should also be considered. In its current version, the application assumes that the patient is able to learn self-management in seven days without considering a possible re-admission and continuation of scoring after discharge. Perspectives on delivery of data between sectors, privacy/security and integration with the electronic health record should all be of major concern in the upcoming iterations. As mentioned in Chapter \ref{chap:problemanalysis}, continuity in the care is an important parameter that should not be underestimated, when a patient is discharged from the hospital. 

In terms of this project, more focus should be on designing the  application to support nurses and influence their health choices for improved care based on existing knowledge about stoma patients. The application should provide guidelines and recommendations to the nurse based on existing patient data in the system, which requires more research on the topic. 





%Suggestions ->
% -> As suggested in Chapter x, pre-operative preparation can impact post-operative scores. 

%In spite of the practical concerns related to the design and implementation of the stoma project, I found that the 


%Future perspectives 
%Considerations when upscaling the project --> Not only 7 days at hospital, but also use at home. Patients and relatives might have other needs, if they have to use this tool. Current version targeted nurses. 

%Refer to image 
%
%Stoma Project as a Tool for Patients
%\begin{itemize}
%\item Patient empowerment 
%\item Continuity of care 
%\end{itemize}
%
%Integration in EPJ 


%Tool makes it easier for nurses , but does it meet the patients' needs for individualized 

%Important to look at, how to make patients accept their stoma before they can be taught self-management skills.



