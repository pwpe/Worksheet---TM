\chapter{Preface}

The following portfolio was developed during Spring 2016 on the tenth semester of Medialogy at Aalborg University. The project is a specialisation in Interaction and will therefore have its main focus on understanding and analysing elements in human centred interaction, using relevant methods to design solutions and implementing an interactive system based on a design solution.

\section*{Reading Guide}
In the portfolio, all the source references are gathered in the \textit{Bibliography} chapter, listed using the Harvard system of referencing. In the body text, a source is cited as [Surname, Year of publication]. The full details of the given source can be found in the reference chapter with the following information: \textit{Author, title and publisher}. Web pages are referenced with: \textit{Author, title and data}.

Figures and tables are numbered in accordance with chapter number. For instance, the first figure in Chapter 4 has the number 4.1., the second figure has the number 4.2., etc. Each figure and table is referred to in the body text and given an explanatory text in addition to the numbering. Abbreviations are introduced in their extended form the first time they appear. Enclosed with this portfolio is a CD containing the paper, portfolio, audio visual production (documenting the project) and other extra materials. 

\section*{Thanks}
Thanks are given to Center for Telemedicin and head of the center Britta Ravn for entering in this project. Thanks to Silkeborg Sygehus and Consultant Frank Andersen, COPD-nurse Sanne B{\o}rgesen and Nurse Trine Juul Volsh{\o}j for professional discussion and making the final evaluation possible by lending us needed equipment. We would also like to thank all patients involved in this project for their big contribution in field studies, design sessions and interviews. 
Finally, we thank our supervisor Hendrik Knoche for continuously good feedback and professional discussion.

\fxnote{sources from fp images}