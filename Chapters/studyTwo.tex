%design session experience
During previous co-designing events researchers have  experienced limited ability among COPD-patients to interact with co-designing tools due to the physical limitations of their disease. While conversing some patients are even not able to participate in other activities \citep{genTech}. Based on these experiences we scaled down our initial idea of the patient as an active partner in a co-designing event. We made a concise plan where patients only were expected to either talk or interact with a prototype. To further reduce the workload on the patients, we gave/sent each patient a workbook with identical assignments for completion over three days prior to the actual design session. The idea with the workbook was to spread out work from the design session on more days, freeing up time for talking and sharing thoughts during the session and sensitize the patients to the topic, promote reflection about collection and reflection. Initially we wanted to develop a prototype with them, but with the co-designing experience from \citep{genTech} we decided to developed a prototype that simulates our proposal of a telehealth system, which the patients could try and criticize. Ideally we wanted to gather all the patients in one room for a common design session but since the patients are less mobile and easily gets tired, the sessions were conducted as individual sessions in the patients own homes. We used the same COPD-patients as in study one, which enabled us to ask some follow-up questions that had arisen since study one. However one patient was hospitalized and could not participate meaning the study included five COPD-patients.

\textbf{The Workbook}

The workbook assignments dealt with collection, data reliability and reflection. We wanted to know more on how patients collect objective and subjective data (both binary and granulated subjective data). We also wanted to know if patients understand the meaning of the values and while collecting reflect on whether the readings are reliable and if they repeat measures. We wanted to see if selected binary subjective measures were based on estimation or recall. Further, we wanted to know if visualizing measures relative to a normal area trigger any reflection and if access to previous subjective measures (both  binary and granularity) affect the assessment. We also included measuring assignments to include their personal measure in the prototype.


\textbf{The Prototype}

The telehealth prototype contains a combined visualisation of the measures with discrete context-related measures as in "Financial Visualization Case Study: Correlating Financial Timeseries and Discrete Events to Support Investment Decisions" and tooltips as recommended in "Investigating time series visualisations to improve the user experience". The prototype did also contain a dashboard for fast (not detailed) visualisations of the data, showing downward or upward trends, which potentially could facilitate cognitive dissonance.


We here bring the activity plan from the study.

\section{Activity Plan}

\textbf{Agenda:}
\begin{itemize}
\item Consent form
\item Part one - Follow-up questions from first visit
\item Part two - Workbook
\item Part three - Prototype 
\end{itemize}

\textit{Process:}

Two researchers were present during each session. The researcher that facilitated study one continued the interview from last time with the follow-up questions while the other researcher prepared part two by looking through the patient's workbook. During part two the first researcher prepared the prototype by adding in personal measures from the patient's workbook in the prototype (sometimes this process was done right before the session if the researchers arrived early to the patient's house).

\textbf{Part 1 - Follow-up}

The follow-up questions are not meant to take up a

Time:
\begin{itemize}
\item How long time have you used Tunstall Healthcare and/or AmbuFlex?%Hvor lang tid har du brugt Tunstall og/eller Ambuflex? Eller en anden telemedicinsk løsning?
\item How long time do you spend on measuring and sending data through  AmbuFlex to the hospital?%Hvor lang tid bruger du på at måle og sende data ind til hospitalet?
\item How much time are you willing to spend on measuring and sending data?%Hvad er din smertegrænse?
\item How long time did it take to learn to use AmbuFlex?%Hvor lang tid har du brugt på at lære at bruge AF?
\item How does the weather influence your COPD (if it influences at all)? %Hvordan påvirker vejret din KOL, hvis den gør?
\end{itemize}

Guidelines:
\begin{itemize}
\item Have you been taught how to do correct measures? (e.g. when to use the saturation device?) %har du fået vejledning i hvordan du tager korrekte målinger?
%Fx. hvornår saturationsmåler skal bruges? (fysisk tilstand)
\end{itemize}

Collection phases:
\begin{itemize}
\item How do you prepare for using the pulse oximeter?%kan du fortælle mig, hvordan du gør dig klar til at foretage dine målinger og hvad du gør? (før)
\item Would you like to have access to your previous measures?%Kunne du tænke dig at have adgang til dine tidligere målinger? 
\end{itemize}

Reflection:
\begin{itemize}
\item Anything specific you need to better reflect on your measures?%Noget specielt der skal til at for at du lettere kan reflektere over dine målinger?
\item Do you miss visualisations of your measures?%savner du visualisering/grafer?
\end{itemize}

In case the patient do parallel tracking:
\begin{itemize}
\item Have you improved your tracking system while tracking? %Har du løbende lavet det om, forbedret det?
\item What do you think of your own tracking system?%Hvad synes du selv om dit eget system?
\item Is there anything you miss?%Er der noget du mangler?
\item Is there something that could be easier?%Er der noget der kunne være lettere?
\end{itemize}

Other:
\begin{itemize}
\item In cases of exacerbation, do you start medication earlier when having telehealth?%Føler du at du får opstartet medicinering tidligere når du bruger TM?
\item What motivates you to use AmbuFlex?%Hvad motiverer dig til at bruge Ambu-Flex?
\end{itemize}

\textbf{Part 2 - Workbook}

- Purpose: Feedback on improved collection and understanding how it fosters reflection while collecting

\begin{enumerate}
\item Discussion on time of measurement (afternoon or morning) - pros/cons
\item Discussion on use of context variables on page 4 + 10
\begin{enumerate}
	\item Identification of level(s) of reflection:
	\begin{enumerate}
		\item When annotating context-related variables in the comment boxes
		\item When checking context-related variables in the check-boxes 
	\end{enumerate}
	\item Discussion on differences between above-mentioned two methods in terms of use, reflection and preference
	\item Discussion on relevance (e.g. cold finger) and priority of variables. 
	\begin{enumerate}
		\item Were you aware that these context variables could influence your measure? Discussion on need for explanation on context-related variables (in guidelines) 
		\item Any new-found variables?
		\item Discussion on shorter versions 
		\item Subjective measure: Dyspnea
	\end{enumerate}
	\item Discussion on what the assessment was based on (recall or estimation). Identification of level(s) of reflection, when assessing:
	\begin{enumerate}
		\item Binary
		\item Granularity (5 ordinal categories)
		\begin{enumerate}
			\item How is it different from binary? Better/worse? Why?
		\end{enumerate}
		\item Numbers (presented in workshop)
		\item Visuals
	\end{enumerate}
	\item Comparison of above-mentioned methods
\end{enumerate}
\item Objective measure: Normal area
\begin{enumerate}
	\item Ask about previous experience with graphs (control)
	\item Identification of level(s) of reflection :
	\begin{enumerate}
		\item What thoughts did the graph trigger?
		\item How are you measurements in relation to the normal area? What do you think of that? What do you think of the normal area shown?
	\end{enumerate}
\end{enumerate}
\item About having a reference graph when assessing binary measures (Identification of level(s) of reflection)
\begin{enumerate}
	\item What thoughts did the graph trigger?
	\item How does it affect your answer to whether you feel more breathless than usual when you can see your previous answers?
\end{enumerate}
\item About having a reference graph when assessing more granular answers (Identification of level(s) of reflection)
\begin{enumerate}
	\item How does it affect your answer to whether you felt breathless today (assessed on 5-point), when you can see your previous answers?
	\item Is there any benefit in seeing your previous measures?
\end{enumerate}
\end{enumerate}


\textbf{Part 3 - Prototype}

- Purpose: Primarily for usability purpose and feedback on visualisation part  

Dashboard:
\begin{itemize}
\item You are on the dashboard. What do you notice first?
\item What does the arrow besides your oxygen saturation measure tell you? What does the arrow besides phlegm tell you?
\item What does the status indicator besides oxygen saturation tell you?
\item How do you feel about the color indications on your health status?
\item Imagine that you want to input a single oxygen saturation measure. Show me what you would do. 
\end{itemize}

Collection: 
\begin{itemize}
\item You want to go back now. Show me what you would do.
\end{itemize}

Dashboard:
\begin{itemize}
\item You want to enter all your measurements. Show me what you would do. (OBS: In this version you will only be able to measure two.)
\end{itemize}

Collection
\begin{itemize}
\item You have taken your saturation measure and it is 84 today. Enter it in the system.
\item You want to take your next measurement. What do you do now?
\item You had extreme shortness of breath today. Enter it in the system.
\item Mark that you have taken your medicine that affects your shortness of breath. Then mark that you have been physically active.
\item You want to save your measurement.
\item Do you want to send your measurement to the hospital? If yes, proceed.
\end{itemize}

Dashboard:
\begin{itemize}
\item Now you are back on the dashboard. *Something about reflective question??*
\item Now you want to see all your previous measures. What do you do?
\end{itemize}

Visualisations:

\textit{\small Visualisation 1}
\begin{itemize}
\item You see the graph for your oxygen saturation. You also want to see dyspnea, what do you do?
\item Instead of dyspnea, you know what to see cough, what do you do?
\item You want to see when stress has affected your oxygen saturation measures. What do you do?
\item What do you think about this visualisation? What do you notice on this visualisation?
\begin{itemize}
 \item Identification of level(s) of reflection
 \end{itemize}
\end{itemize}



\textit{\small Visualisation 2}
\begin{itemize}
\item You see the graph for your oxygen saturation. You also want to see dyspnea, what do you do?
\item You also want to see pulse. What do you do?
\item You want to remove the graph for dyspnea. What do you do?
\item You want to see more details on the oxygen saturation measure made on the 4th of April. What do you do?
\item What do you think about this visualisation? What do you notice on this visualisation?
\begin{itemize}
	\item Identification of level(s) of reflection
	\end{itemize}
\end{itemize}

\textit{\small Visualisation 3}
\begin{itemize}
\item You see the graph for your oxygen saturation. You also want to see dyspnea, what do you do?
\item You want to see whether physical activity before the measure has influenced your oxygen saturation measure. What do you do? 
\item What do you think about this visualisation? What do you notice on this visualisation?
\item Identification of level(s) of reflection
\end{itemize}

\textit{\small Visualisation 4}
\begin{itemize}
\item You see the graph for your oxygen saturation. You also want to see dyspnea, what do you do?
\item You want to see more details on the oxygen saturation measure made on the 4th of April. What do you do?
\item What do you think about this visualisation? What do you notice on this visualisation?
\item Identification of level(s) of reflection
\end{itemize}

Other questions
\begin{itemize}
\item Visualisation of weather. How does weather affect you?
\item Currently, you share your measures with the hospital. Imagine a system, where that is not required. Can you imagine that you entered measurements into the system, but did not want to share with the hospital? Why/why not?
\end{itemize}