A stoma nurse has an important role in educating a patient to adjust to the new lifestyle after a stome surgery by teaching the patient self-management skills, also called stoma self-care. Stoma self-care is acknowledged as a crucial factor in determining a patient's quality of life after a surgery. The Urostomy Education Scale (UES) is a standardized educational intervention that aims to facilitate the nurses in teaching patients stoma self-care. It is intended to improve quality of stoma care and reduce random clinical practice. Despite of the many advantages of the UES, the current implementation in practice does not support nurses in teaching stoma self-care to patients using the UES. Challenges include missing overview of patient's progression, portability and availability of the patient's data. In this project, we designed and implemented an application to tablet devices that based on the UES facilitates the nurse in teaching the patients self-management skills. Through an iterative design and prototyping process, we found that mobile technology opens up for new perspectives to improve quality of care for stoma patients.  