Telehealth systems for early detection and treatment of chronic conditions have seen increased use. But the effects on user needs and concerns when healthcare provider continuously monitor and patients provide subjective and objective data over time is poorly understood. Personal Informatics literature informed the analysis of interviews with six Chronic Obstructive Pulmonary Disease (COPD) patients to improve understanding of user needs and concerns in the use of a state of the art telehealth solution. While patients generally felt taken care of, the system in many ways did not meet user needs, e.g. due to difficulties assessing reliable subjective measures and no support on reflection and follow-up action. Interviews, workbooks and design feedback sessions with patients served as the foundation for redesigning the system to support data collection and reflection. Findings from a two week trial involving five COPD patients showed that the system supported one of two types of patients in becoming more informed and aware about their health status, leading to increased empowerment in their everyday life and motivation to set goals and improve condition.